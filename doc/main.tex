\documentclass[twocolumn,11pt]{article}
\usepackage{fullpage}
\usepackage{hyperref}
\usepackage{amsmath}
\usepackage{amssymb}
\usepackage{amsthm}
\usepackage{fancybox}
\usepackage{graphicx}
\usepackage{caption}
\usepackage{subcaption}
\usepackage{float}
\usepackage{natbib}
\bibliographystyle{abbrv}

\begin{document}

\title{Pandora: microbial characterization of human tissue from RNAseq}

\author{Sakellarios Zairis$^*$, Francesco Abate$^*$, Oliver Elliott, and Raul Rabadan\\
\\
Department of Systems Biology, Columbia University}
\maketitle


%%%%%%%%%%%%%%%%%%%%%%%%%%%%%%%%%%%%%%%%%%%%%%%%%%%%%%%%%%%%%%%%%%%%%%%%%%%%%%%%%%%%%%%%%%%%%%%%%%%%
%%%%%%%%%%%%%%%%%%%%%%%%%%%%%%%%%%%%%%%%%%%%%%%%%%%%%%%%%%%%%%%%%%%%%%%%%%%%%%%%%%%%%%%%%%%%%%%%%%%%

\begin{abstract}

Whole transcriptome sequencing (RNAseq) of human tissue has traditionally been used for gene expression profiling.
The compartment of transcripts not deriving from the human genome, however, offers a limited window into host-microbe interactions and local tissue conditions.
Microbial detection from RNAseq typically follows the strategy of \textit{in silico} host subtraction, followed by a combination of database local alignment and \textit{de novo} assembly.
One step beyond detection, however, lie the issues of annotating the functional significance of non-human transcripts and quantifying host-microbe interactions.
We implement an open-source pipeline, for nominating microbial taxa whose concentration in a series of tissue samples carries an interpretable correlate in the host expression profile.
Active viral replication detected in tumor samples can point to local suppression of immune surveillance, or identify infection with a known oncovirus.
Cellular microbe populations nominated by Pandora can indicate global immunocompromise or inform anti-microbial regimens, while bacterial oxygen preferences can help identify hypoxia in tumor samples prior to a therapy whose efficacy is a strong function of oxygen tension. 

\end{abstract}

%%%%%%%%%%%%%%%%%%%%%%%%%%%%%%%%%%%%%%%%%%%%%%%%%%%%%%%%%%%%%%%%%%%%%%%%%%%%%%%%%%%%%%%%%%%%%%%%%%%%
%%%%%%%%%%%%%%%%%%%%%%%%%%%%%%%%%%%%%%%%%%%%%%%%%%%%%%%%%%%%%%%%%%%%%%%%%%%%%%%%%%%%%%%%%%%%%%%%%%%%

\section{Introduction}

For over half a century it has been known that bacterial spores from obligate anaerobes can migrate to and germinate within hypoxic regions of mammalian tumors.~\cite{malmgren1955localization}

%%%%%%%%%%%%%%%%%%%%%%%%%%%%%%%%%%%%%%%%%%%%%%%%%%%%%%%%%%%%%%%%%%%%%%%%%%%%%%%%%%%%%%%%%%%%%%%%%%%%
%%%%%%%%%%%%%%%%%%%%%%%%%%%%%%%%%%%%%%%%%%%%%%%%%%%%%%%%%%%%%%%%%%%%%%%%%%%%%%%%%%%%%%%%%%%%%%%%%%%%

\section{Methods}

For over half a century it has been known that bacterial spores from obligate anaerobes can migrate to and germinate within hypoxic regions of mammalian tumors.~\cite{malmgren1955localization}

%%%%%%%%%%%%%%%%%%%%%%%%%%%%%%%%%%%%%%%%%%%%%%%%%%%%%%%%%%%%%%%%%%%%%%%%%%%%%%%%%%%%%%%%%%%%%%%%%%%%
%%%%%%%%%%%%%%%%%%%%%%%%%%%%%%%%%%%%%%%%%%%%%%%%%%%%%%%%%%%%%%%%%%%%%%%%%%%%%%%%%%%%%%%%%%%%%%%%%%%%

\section{Results}

For over half a century it has been known that bacterial spores from obligate anaerobes can migrate to and germinate within hypoxic regions of mammalian tumors.~\cite{malmgren1955localization}

%%%%%%%%%%%%%%%%%%%%%%%%%%%%%%%%%%%%%%%%%%%%%%%%%%%%%%%%%%%%%%%%%%%%%%%%%%%%%%%%%%%%%%%%%%%%%%%%%%%%
%%%%%%%%%%%%%%%%%%%%%%%%%%%%%%%%%%%%%%%%%%%%%%%%%%%%%%%%%%%%%%%%%%%%%%%%%%%%%%%%%%%%%%%%%%%%%%%%%%%%

\section{Discussion}

For over half a century it has been known that bacterial spores from obligate anaerobes can migrate to and germinate within hypoxic regions of mammalian tumors.~\cite{malmgren1955localization}

%%%%%%%%%%%%%%%%%%%%%%%%%%%%%%%%%%%%%%%%%%%%%%%%%%%%%%%%%%%%%%%%%%%%%%%%%%%%%%%%%%%%%%%%%%%%%%%%%%%%
%%%%%%%%%%%%%%%%%%%%%%%%%%%%%%%%%%%%%%%%%%%%%%%%%%%%%%%%%%%%%%%%%%%%%%%%%%%%%%%%%%%%%%%%%%%%%%%%%%%%

\subsection*{Acknowledgments}

This work is supported by a TL1 personalized medicine fellowship (5TL1TR000082), and NIH grants (R01 CA179044, R01CA185486, R01GM117591, U54 CA193313).

%%%%%%%%%%%%%%%%%%%%%%%%%%%%%%%%%%%%%%%%%%%%%%%%%%%%%%%%%%%%%%%%%%%%%%%%%%%%%%%%%%%%%%%%%%%%%%%%%%%%
%%%%%%%%%%%%%%%%%%%%%%%%%%%%%%%%%%%%%%%%%%%%%%%%%%%%%%%%%%%%%%%%%%%%%%%%%%%%%%%%%%%%%%%%%%%%%%%%%%%%

\bibliography{refs}

\end{document}
